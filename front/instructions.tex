\clearpage
\thispagestyle{empty}

\begin{center}
    \heiti\xiaoer\textbf{说~~~~~~明}
\end{center}
\setlength{\parskip}{7.5bp}
\begin{spacing}{1.25}
\heiti\xiaosan\textbf{一、开题报告主要内容}


\setlength{\parskip}{7.4bp}
\heiti\sihao{1.课题来源及研究的目的和意义}

\songti\xiaosi{(正文  宋体小4号字,行距1.25倍,段前0行,段后0行)}

\heiti\sihao{
    2.国内外在该方向的研究现状及分析
    
    3.主要研究内容
    
    4.研究方案
    
    5.进度安排,预期达到的目标
    
    6.课题已具备和所需的条件、经费
    
    7.研究过程中可能遇到的困难和问题,解决的措施
    
    8.主要参考文献
}
\setlength{\parskip}{7.4bp}


\heiti\xiaosan\textbf{二、开题报告要求}

\setlength{\parskip}{7.4bp}
\heiti\sihao{
    1.开题报告的字数应在3000字以上。

    2.参考文献的要求:
}
\setlength{\parskip}{7.4bp}


\setlength{\parskip}{0bp}
\songti\xiaosi{
(1)理工类论文的参考文献一般为10-15篇,其中学术期刊类文献不少于7篇,外文文献不少于3篇(特殊专业可酌情确定明确要求,并报教务部备案);文科、管理类论文,参考文献一般为15-20篇,其中学术期刊类文献不少于12篇,外文文献不少于3篇。近五年的文献数不应少于总数的1/3,应有近两年的参考文献。教材、产品说明书、国家标准、未公开发表的研究报告不宜作为参考资料。

(2)参考文献按在开题报告中出现的次序列出。

(3)参考文献书写顺序:序号 作者.文章名.学术刊物名.年,卷(期):引用起止页。
}
\setlength{\parskip}{0bp}
\end{spacing}
\setlength{\parskip}{0bp}
% Local Variables:
% TeX-master: "../mainart"
% TeX-engine: xetex
% End:
